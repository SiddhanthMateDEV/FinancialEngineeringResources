\documentclass{article}
\usepackage{amsmath} % Used for advanced math commands
\usepackage{amssymb} % Used for additional math symbols
\usepackage{amsthm}  % Optional: Used for theorem environments
\usepackage{bm}      % Used for bold math symbols

\begin{document}

\section*{Option Premium Calculation}

To calculate the option premium, we use the Black-Scholes formula. The parameters are defined as follows:

\begin{itemize}
    \item \( S \): Spot price
    \item \( K \): Strike price
    \item \( T \): Time to maturity (in years)
    \item \( r \): Risk-free rate of interest
    \item \( \sigma \): Volatility of the underlying asset
    \item \( \Phi \): Cumulative distribution function of the standard normal distribution
\end{itemize}

\subsection*{Formulas}

\begin{align*}
d_1 & = \frac{\ln\left(\frac{S}{K}\right) + \left(r + \frac{\sigma^2}{2}\right)T}{\sigma \sqrt{T}} \\
d_2 & = d_1 - \sigma \sqrt{T}
\end{align*}

\subsection*{Call Option Premium}

For a call option (denoted as 'CE' or 'c'), the premium is given by:

\[
C = S \Phi(d_1) - K e^{-rT} \Phi(d_2)
\]

\subsection*{Put Option Premium}

For a put option (denoted as 'PE' or 'p'), the premium is given by:

\[
P = K e^{-rT} \Phi(-d_2) - S \Phi(-d_1)
\]

\subsection*{Conditions}

\begin{itemize}
    \item If the option type is call ('CE' or 'c'), use the call option premium formula.
    \item If the option type is put ('PE' or 'p'), use the put option premium formula.
    \item If the option type is invalid, raise an error.
\end{itemize}

\end{document}
